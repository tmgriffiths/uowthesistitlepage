%------------------------Thesis Master------------------------------------------------------------%
% Thomas Griffiths 2013-07-16 tmg994@uowmail.edu.au                                               %
% If you get stuck read my comments here and in the preamble (thesispreamble.tex)                 %
% hopefully they can help you find where your answers will be.                                    %
% I highly reccomend making friends with Google, it knows quite a bit.                            %
%-------------------------------------------------------------------------------------------------%

\documentclass[12pt,oneside,english,]{book}

%------------------------ Preamble and bibliography resources
%---------------UOW Thesis Preamble-----------------------------------------
% Thomas Griffiths 2013-07-16 tmg994@uowmail.edu.au
%
% I encourage you to read the documentation for each of the packages below. 
% They contain instructions for implementation, and examples of their use. 
% If you get stuck read my comments, hopefully they can help you find where 
% your answers will be. I highly reccomend making friends with señor Google, 
% he knows quite a bit. The wikibook on LaTeX is also very helpful. 
% http://en.wikibooks.org/wiki/LaTeX/
% The packages i've loaded here centre around a thesis written for physical 
% chemistry. As such there might not be a package that has the capabilities 
% you are looking for. I encourage you to look on ctan, www.ctan.org/‎,  
% for packages that might be relevant to your degree.
%
%---------------------------------------------------------------------------

\usepackage{geometry}
	\geometry{a4paper,inner=3.8cm, outer=2cm, top=3cm, bottom=2cm}
	% Dimensions from UOW thesis guidelines.
	\pdfpagewidth=\paperwidth 
	\pdfpageheight=\paperheight
	% This acts as a failsafe to ensure things aren't stretched or moved when it's finally printed as a PDF.

%\usepackage[british,australian]{babel} % sets the language localisation. There's no real point as the default is english

%\usepackage[parfill]{parskip} % Activate to begin paragraphs with an empty (return) line 
    \setlength{\parindent}{4ex}	% Sets the length of the paragraph indent. Current setup has a an indent. Disable this if you activate return line above.
	
\usepackage{setspace}
	\doublespace
% Double spacing.
	
\usepackage{graphicx}
	\DeclareGraphicsRule{.tif}{png}{.png}{`convert #1 `dirname #1`/`basename #1 .tif`.png}
% Graphics. Remove me and you won't have any figures, and that would be very boring.

\usepackage{epstopdf} 
% Converts eps to pdf files. Use if you have EPS graphics and you are compiling with PDFTeX. (Fine to leave if you use TeX + DVI then DIV to PDf converter.)

\usepackage{color}
\usepackage[usenames,dvipsnames,svgnames,table]{xcolor}
% Adds the ability to make coloured text and lines throughout the document. See documentation for xcolor.


%--------------------Tables, figures and captions
\usepackage[font={small},labelfont={bf},margin=4ex]{caption}
% Makes bold labeled and smaller font captions. Must be loaded before the longtable package to avoid conflicts! 

\usepackage{longtable} 
% Long tables (more than one page). Different headers and footers for beginning and end pages, etc.

\usepackage{booktabs} 
% Nice looking tables and lines in tables

\usepackage{afterpage} 
% Make a longtable start on the next clear page, but fills the previous one with text first (no random gaps in the text-from long tables anymore! Man, the day I discovered this...)

\usepackage{multirow} 
% Entries in tables over multiple rows

\usepackage{lscape} 
% Pages in landscape

\usepackage{pdflscape} 
% Landscape pages also rotated in the pdf

\usepackage{wrapfig} 
% Allows figures that don't take up the entire width of the page, wrapping the text around the figure

\usepackage[position=top,singlelinecheck=false,captionskip=4pt]{subfig} 
% Multiple figures in an individual figure. Fig. 1 a, b, c, etc. each with, or without, it's own individual caption, and with a global caption for all sub figures.

\usepackage{rotating} 
% Allows you to rotate an object by an arbitrary amount

%--------------------Special symbols and fonts
\usepackage{lmodern}
% Sets the font to latin modern rather than computer modern. Required for the lettrine package.

\usepackage{amssymb} 
% Maths symbols

%--------------------Chemistry and Science
\usepackage[version=3]{mhchem} 									
% Chemistry typing. Ridiculous time savings when typing out chemical formulae or equations. Whole equations can be written with this package, for example \ce{CaCO3 + 2HCl -> CaCl2 + H20 + CO2}. You can also add reagents above and below the arrows. READ the documentation, it's awesome.

\usepackage{psfrag} 											
% Replace text in eps files. Ensures identical font formatting for axes and other text things in your figures. You need to be using eps figures for this to work. While it's awesome it can be a lot of work it's used in conjunction with chemstyle below.

\usepackage[bpchem]{chemstyle} 									
% Chemistry formatting and auto numbering of compounds in schemes and figures, you need to use TeX + DVI with eps graphics for this to work. No JPG or PNG here. If you use PDFTex nothing will happen, it'll probably get stuck on a command, but it won't do anything brash. You won't get any sweet auto numbering goodness though.

\usepackage{chemfig}											
% Writing complex chemical structures in text e.g. equation arrays.

\usepackage{siunitx}
	\DeclareSIUnit{\molecule}{molecule}
	\DeclareSIUnit{\day}{day}
	\DeclareSIUnit{\atmosphere}{atm}
	\DeclareSIUnit{\calorie}{cal}
	\DeclareSIUnit{\dalton}{Da}
	%\DeclareSIUnit{\Calorie}{Cal}
% SI units, latin phrases etc. Ensures consitent typesetting of units and numbers etc, but only if you use it. E.g.\SI{256}{}\nano\metre} gives 256 nm, with an unbreaking space so that you dont get a value on one line with the units on the next line.

%--------------------Document sections, headers, footers, and bibliography
\usepackage[header,toc,page]{appendix}								
% Adds the word appendix to the header in the appendix section, adds an appendices entry (like a part) to the toc before listing the appendices.

\usepackage{fancyhdr}											
% for creating different headers and footers

\usepackage{extramarks} 										
% for extrmarks in the longtable page style header and footer. In case you want to have text in the header and footer saying, hey you're on a longtable page in the chapter etc.

%--------------------Bibliography
\usepackage[backend=biber,articletitle=true,style=chem-rsc,doi=false]{biblatex}
% This is the package that lets you create a bibliography. I recommend reading the biblatex documentation to understand all the options i've specified here. BibLaTeX was created to replace BibTeX. It has lots of extra fields and options. I'm also using the biber backend here rather than the default, it copes with unicode and so can deal with accented characters easily.

% Currently this is set up to use RSC style references with article titles displayed.

% Traditionally you would use BibTeX, the newer biblatex is a more powerful bibliogrpy management tool for LaTeX. You can make multiple chapter based bibliographies, footnote bibliographes, sort your references by date, author, order cited, essentially by any bit of citation data you happen to have. You can also have a seperate library with a differnet format for say books and articles. or if you're a PhD student, the thesis references and your publications.

%\usepackage[numbers,super,comma,sort&compress]{natbib}				
	%\setcitestyle{square}										
	% places citations in square brackets to helps to distinguish between powers and citations
%This is the old natbib package that meshes with bibtex (rather than using the newer biblatex). It's here mainly for legacy purposes. Try to shift to biblatex if you can, it is cleaner in it's implementation and creating a custom citation style is easier then with bibtex.

\usepackage[unicode=true,colorlinks=true,linkcolor=black,citecolor=black,urlcolor=black,breaklinks=true]{hyperref}
% The hyperref package allows you to have clickable links in your pdf. It also allows you to have the mailto link associated with your name. It can be  a bit finicky, so load it last.

\usepackage{uowthesistitlepage}
% Creates the title page in accordance with UOW guidelines, includes the definition of the extra fields in \maketitle

%--------------------Command renewals, New commands etc.
\renewcommand{\thefootnote}{\alph{footnote}}							
%letters for footnotes instead of numbers to avoid confusion with references.

%\setcounter{tocdepth}{2}
% Sets the depth of the Table of contents. Currently 2, i.e. subsections.




\addbibresource{Your_Bibliography.bib}

%------------------------Main Document--------------------------
\begin{document}
    \onehalfspace
	
%-------------- Title Page
% Title page info (see uowthesistitlepage package)
\title{A Pretty Swish Title} 
\author{Average J. Blow} % Full name, and any degrees held.
\date{January 1901} % Month Year, alternatively use the \today macro for Month dd, yyyy.
\degree{That Degree You're Studying} % Write it in full: e.g. Bachelor of Science Medicinal Chemistry Honours
\supervisor[2]{Dr. J Bloggs \& Professor J. Citizen}% The optional argument (default 1) in square brackets is the number of supervisors. In the Curly braces list your supervisor(s) seperated by commas.
\school{Your School} % e.g Chemistry

%-------------- Front Matter
\pagestyle{fancy} \frontmatter
\maketitle
\declaration
\addcontentsline{toc}{chapter}{Abstract}
\chapter*{Abstract} % Starred chapter=chapter with no number.
\include{./frontmatter/acknowledgments}

\cleardoublepage
\phantomsection \pdfbookmark[0]{Contents}{Contents} % These \phantomsection are to ensure that the hyperref package hyperlinks to the correct page. If you turn hyperref off they don't do anything so they can just stay here
\tableofcontents

% A list of Figures, uncomment to activate.
%\cleardoublepage
%\phantomsection \label{listoffigures}\addcontentsline{toc}{chapter}{List of Figures}
%\listoffigures

% A list of Schemes, uncomment to activate.
%\cleardoublepage
%\phantomsection \label{lisrofschemes}\addcontentsline{toc}{chapter}{List of Schemes} 
%\listofschemes

% A list of Tables, uncomment to activate.						
%\cleardoublepage
%\phantomsection \label{listoftables}\addcontentsline{toc}{chapter}{List of Tables} 
%\listoftables

%-------------- Chapters
\mainmatter
\pagestyle{fancy}
\chapter{Introduction}
\section{The Title of Your First Section}
\subsection{A Subsection --- for Clarity}
Lorem ipsum dolor sit amet, consectetur adipiscing elit.\cite{pericles} Integer ut congue lectus. Nullam dapibus scelerisque diam, ac convallis dolor convallis non. Sed nec lectus nec sapien interdum commodo nec quis elit. Sed neque augue, pulvinar id imperdiet id, ullamcorper sed sem. Duis pulvinar blandit erat, quis suscipit erat venenatis sit amet. Aenean varius aliquam dignissim. Sed tempus consequat sapien et bibendum. Class aptent taciti sociosqu ad litora torquent per conubia nostra, per inceptos himenaeos. Maecenas arcu ligula, tincidunt non volutpat nec, luctus ac justo. Donec euismod egestas leo, vel convallis mi accumsan sed. Aliquam erat volutpat. Maecenas volutpat lacinia justo, dictum iaculis enim scelerisque in. Vestibulum consequat augue in nisl luctus eget ultrices sem blandit. Donec mi risus, rutrum at tempus in, dapibus vel est. Sed eu ullamcorper velit.\cite{pericles,thekingsenglish}

Duis malesuada ultrices rutrum. In posuere sem dapibus urna accumsan id sollicitudin turpis iaculis. Sed fermentum metus vel dui vulputate consectetur. Nunc nec ante nisi, ac ultricies metus. Nullam sed dui vitae metus vehicula adipiscing. Nunc facilisis tortor neque. Praesent ac nulla at odio lacinia egestas. Nam facilisis vehicula pretium.

Fusce tempus libero vitae leo cursus cursus. Suspendisse iaculis dignissim placerat. Vivamus ornare, lectus in placerat cursus, quam ligula congue risus, a facilisis leo est ut massa. Nulla in metus id eros rutrum aliquam. Cras id velit ut enim iaculis adipiscing facilisis sit amet lorem. Ut id neque id turpis facilisis tempus sit amet nec urna.\cite{bioluminescence:ch1} Ut volutpat magna et dolor elementum eget elementum lorem eleifend. Pellentesque mollis convallis est a tincidunt. Nunc egestas, ante a blandit iaculis, risus orci auctor risus, in pellentesque est sapien id mi. Nunc faucibus porttitor tincidunt.\cite{atkins_inorgchem,atkins_physchem} Fusce eu faucibus dolor. Mauris quis arcu metus, at malesuada lacus. Nam lectus lacus, sodales nec tempus eu, pretium id risus. Cras dignissim aliquet laoreet. Morbi sed arcu id tellus tempor consectetur.\cite{bioluminescence:apA,hori1973}

\subsection{Another Subsection}
Aenean convallis, ante quis convallis viverra, urna lorem venenatis odio, ut bibendum nisi mauris a mi. Nam laoreet arcu a eros dignissim elementum. Class aptent taciti sociosqu ad litora torquent per conubia nostra, per inceptos himenaeos. Proin vitae magna ligula, id consequat felis. Integer rutrum ante vitae tellus convallis volutpat in vitae odio. Vivamus eu turpis et turpis euismod faucibus. Suspendisse at turpis sit amet magna auctor fringilla. Vestibulum quis nulla sed dui pellentesque dapibus.\footnote{This is a footnote. Footnotes are notes at the foot of the page. Literary style guides (and some supervisors) recommend limited use of footnotes; but, publishers often encourage them. Use them as you will but don't go overboard.} Cum sociis natoque penatibus et magnis dis parturient montes, nascetur ridiculus mus. Maecenas id odio nisl. Etiam facilisis elit in urna hendrerit ac iaculis mauris lacinia. Fusce congue ultricies nulla vel elementum. Phasellus suscipit vestibulum tortor at semper. Etiam eu sagittis augue.\cite{bioluminescence,clayden_orgchem}
\begin{figure}
\centering
\includegraphics[height=4cm]{figures/example.jpg}
\caption[Chemistry cat.]{This is chemistry cat. Here he serves to demonstrate a figure in a LaTeX document, complete with caption. Notice his suave bow tie, but also that he has forgotten to label his solutions.}
\end{figure}
Duis malesuada ultrices rutrum. In posuere sem dapibus urna accumsan id sollicitudin turpis iaculis. Sed fermentum metus vel dui vulputate consectetur. Nunc nec ante nisi, ac ultricies metus. Nullam sed dui vitae metus vehicula adipiscing. Nunc facilisis tortor neque. Praesent ac nulla at odio lacinia egestas. Nam facilisis vehicula pretium.

Praesent sed diam arcu, quis aliquam libero. Pellentesque a nulla vulputate lectus rhoncus vulputate ut in urna. Quisque sapien sem, convallis ac porta sit amet, aliquet sit amet eros. Etiam quis lorem ligula. Mauris elit sapien, ultrices nec vulputate ut, porta eu leo. Ut consequat accumsan commodo. Sed ut ultricies ante.

Integer egestas pharetra sem, ut viverra sapien. Suspendisse dapibus et nisl et viverra. Fusce dignissim sagittis sapien, eu molestie sapien fermentum nec. Praesent tristique convallis tortor, vitae molestie libero interdum facilisis. Integer nec dignissim sem, ac semper nulla. Integer tellus nunc, rutrum eu placerat eu, semper sit amet felis. Fusce vitae leo at ipsum aliquet adipiscing. Etiam magna nunc, elementum non dolor ac, laoreet gravida arcu. Nam congue nisi a tortor hendrerit, vitae ultrices lacus dignissim. Phasellus sit amet tristique nisi.

\begin{table}[ht]
  \centering
  \caption{Add caption}\label{tab:addlabel}
    {\small
    \begin{tabular}{*{2}{r@{-}lr@{-}lc}}
    \toprule
    \multicolumn{2}{c}{\textbf{Substrate Atom}} & \multicolumn{2}{c}{\textbf{Protein Atom}} & \textbf{Occ.} &\multicolumn{2}{c}{\textbf{Substrate Atom}} & \multicolumn{2}{c}{\textbf{Protein Atom}} & \textbf{Occ.} \\
    \midrule
        Mol1a & N    & Tyr138   & HH     & 0.13 & Mol1b & N    & Tyr138   & HH     & 0.15 \\
              & O    & Trp179   & HE1    & 0.28 &       & O    & His175   & HE2    & 0.33 \\
              & O1   & Trp92    & HE1    & 0.18 &       & O1   & Trp92    & HE1    & 0.40 \\
              & H4   & His22    & ND1    & 0.68 &       & H3   & His22    & ND1    & 1.00 \\
              & O2   & His175   & HE2    & 0.33 &       & O2   & Tyr138   & HH     & 0.27 \\
              & O2   & Tyr190   & HH     & 0.15 &       & O2   & His175   & HE2    & 0.47 \\
              & H16  & Ile111   & O      & 0.11 &       &      &          &        &      \\
              & H16  & Tyr190   & OH     & 0.72 &       &      &          &        &      \\
    \bottomrule
    \end{tabular}%
    }%
\end{table}

Ut egestas tortor sit amet pulvinar elementum. Nulla facilisi. Vivamus non elit magna. In rhoncus cursus tortor vel fringilla. Duis in tortor sed libero imperdiet bibendum ut at magna. Suspendisse ipsum enim, scelerisque non dolor fermentum, commodo ultrices sapien. Nunc nec justo aliquam, elementum enim ut, feugiat dui. Nullam placerat dui sollicitudin posuere sollicitudin. Aliquam iaculis purus a adipiscing auctor. Aliquam eu faucibus mi. Donec fringilla ultricies posuere. In sed porttitor risus. Etiam aliquam ornare felis, pulvinar auctor neque mollis et. Pellentesque habitant morbi tristique senectus et netus et malesuada fames ac turpis egestas. Mauris rutrum felis quis pellentesque tincidunt. Proin at feugiat turpis. Maecenas eget ligula quis odio elementum iaculis ac nec est. Nam sit amet dolor risus. Morbi dictum, dui a rutrum dapibus, odio orci commodo turpis, sit amet luctus neque nibh imperdiet mauris. Pellentesque venenatis facilisis orci at malesuada. Fusce ultrices ultrices nibh, in lacinia augue. In ut sapien ornare, vestibulum ligula nec, rutrum lacus.

\begin{equation}
    W(\boldsymbol{r})= \beta^{-1} \ln \sum_{m=1}^M\sum_{k=1}^C\frac{e^{-\beta\Delta E(\boldsymbol{r},\boldsymbol{q_m},\boldsymbol{\Omega_k})}}{MC}
\end{equation}

Aliquam orci quam, tincidunt nec nisi euismod, dictum rhoncus leo. Maecenas sit amet aliquet massa. Praesent eu elementum mauris. Phasellus tincidunt augue sit amet metus tristique vulputate. Duis auctor eu velit et pharetra. Ut ut tortor et mi imperdiet rutrum. Nam pellentesque felis eget elit ultrices, at vulputate sem rhoncus. Curabitur adipiscing sapien velit, eget lobortis velit ullamcorper ut. Donec et tempor nulla. Proin ut velit sed erat porttitor pellentesque vitae non arcu. Mauris ut justo non risus commodo pretium.

Morbi sed sapien in nisi facilisis pellentesque. Nunc id arcu consectetur magna commodo posuere. Quisque malesuada quis libero nec facilisis. Cras pharetra sem ac ipsum faucibus, eu molestie erat fermentum. Proin ac magna tincidunt nisl tempus iaculis id ac eros. Fusce ullamcorper porttitor purus. Cras iaculis ullamcorper lacus vitae dapibus. Vestibulum nec suscipit risus. Donec quis leo vitae mi facilisis convallis. Pellentesque in ipsum nisi.Proin et placerat augue. Curabitur imperdiet tellus scelerisque, dapibus lectus nec, aliquet tortor. Etiam dignissim ornare orci, nec condimentum lectus. Praesent dictum nisl semper dui semper adipiscing. Ut in diam neque. Cras rutrum erat et nunc ultrices commodo at eget diam. Maecenas adipiscing tincidunt rutrum.

Integer egestas pharetra sem, ut viverra sapien. Suspendisse dapibus et nisl et viverra. Fusce dignissim sagittis sapien, eu molestie sapien fermentum nec. Praesent tristique convallis tortor, vitae molestie libero interdum facilisis. Integer nec dignissim sem, ac semper nulla. Integer tellus nunc, rutrum eu placerat eu, semper sit amet felis. Fusce vitae leo at ipsum aliquet adipiscing. Etiam magna nunc, elementum non dolor ac, laoreet gravida arcu. Nam congue nisi a tortor hendrerit, vitae ultrices lacus dignissim. Phasellus sit amet tristique nisi.

Ut egestas tortor sit amet pulvinar elementum. Nulla facilisi. Vivamus non elit magna. In rhoncus cursus tortor vel fringilla. Duis in tortor sed libero imperdiet bibendum ut at magna. Suspendisse ipsum enim, scelerisque non dolor fermentum, commodo ultrices sapien. Nunc nec justo aliquam, elementum enim ut, feugiat dui. Nullam placerat dui sollicitudin posuere sollicitudin. Aliquam iaculis purus a adipiscing auctor. Aliquam eu faucibus mi. Donec fringilla ultricies posuere. In sed porttitor risus. Etiam aliquam ornare felis, pulvinar auctor neque mollis et. Pellentesque habitant morbi tristique senectus et netus et malesuada fames ac turpis egestas. Mauris rutrum felis quis pellentesque tincidunt. Proin at feugiat turpis. Maecenas eget ligula quis odio elementum iaculis ac nec est. Nam sit amet dolor risus. Morbi dictum, dui a rutrum dapibus, odio orci commodo turpis, sit amet luctus neque nibh imperdiet mauris. Pellentesque venenatis facilisis orci at malesuada. Fusce ultrices ultrices nibh, in lacinia augue. In ut sapien ornare, vestibulum ligula nec, rutrum lacus.

Aliquam orci quam, tincidunt nec nisi euismod, dictum rhoncus leo. Maecenas sit amet aliquet massa. Praesent eu elementum mauris. Phasellus tincidunt augue sit amet metus tristique vulputate. Duis auctor eu velit et pharetra. Ut ut tortor et mi imperdiet rutrum. Nam pellentesque felis eget elit ultrices, at vulputate sem rhoncus. Curabitur adipiscing sapien velit, eget lobortis velit ullamcorper ut. Donec et tempor nulla. Proin ut velit sed erat porttitor pellentesque vitae non arcu. Mauris ut justo non risus commodo pretium.

Morbi sed sapien in nisi facilisis pellentesque. Nunc id arcu consectetur magna commodo posuere. Quisque malesuada quis libero nec facilisis. Cras pharetra sem ac ipsum faucibus, eu molestie erat fermentum. Proin ac magna tincidunt nisl tempus iaculis id ac eros. Fusce ullamcorper porttitor purus. Cras iaculis ullamcorper lacus vitae dapibus. Vestibulum nec suscipit risus. Donec quis leo vitae mi facilisis convallis. Pellentesque in ipsum nisi.Proin et placerat augue. Curabitur imperdiet tellus scelerisque, dapibus lectus nec, aliquet tortor. Etiam dignissim ornare orci, nec condimentum lectus. Praesent dictum nisl semper dui semper adipiscing. Ut in diam neque. Cras rutrum erat et nunc ultrices commodo at eget diam. Maecenas adipiscing tincidunt rutrum.

Integer egestas pharetra sem, ut viverra sapien. Suspendisse dapibus et nisl et viverra. Fusce dignissim sagittis sapien, eu molestie sapien fermentum nec. Praesent tristique convallis tortor, vitae molestie libero interdum facilisis. Integer nec dignissim sem, ac semper nulla. Integer tellus nunc, rutrum eu placerat eu, semper sit amet felis. Fusce vitae leo at ipsum aliquet adipiscing. Etiam magna nunc, elementum non dolor ac, laoreet gravida arcu. Nam congue nisi a tortor hendrerit, vitae ultrices lacus dignissim. Phasellus sit amet tristique nisi.

Ut egestas tortor sit amet pulvinar elementum. Nulla facilisi. Vivamus non elit magna. In rhoncus cursus tortor vel fringilla. Duis in tortor sed libero imperdiet bibendum ut at magna. Suspendisse ipsum enim, scelerisque non dolor fermentum, commodo ultrices sapien. Nunc nec justo aliquam, elementum enim ut, feugiat dui. Nullam placerat dui sollicitudin posuere sollicitudin. Aliquam iaculis purus a adipiscing auctor. Aliquam eu faucibus mi. Donec fringilla ultricies posuere. In sed porttitor risus. Etiam aliquam ornare felis, pulvinar auctor neque mollis et. Pellentesque habitant morbi tristique senectus et netus et malesuada fames ac turpis egestas. Mauris rutrum felis quis pellentesque tincidunt. Proin at feugiat turpis. Maecenas eget ligula quis odio elementum iaculis ac nec est. Nam sit amet dolor risus. Morbi dictum, dui a rutrum dapibus, odio orci commodo turpis, sit amet luctus neque nibh imperdiet mauris. Pellentesque venenatis facilisis orci at malesuada. Fusce ultrices ultrices nibh, in lacinia augue. In ut sapien ornare, vestibulum ligula nec, rutrum lacus.

Aliquam orci quam, tincidunt nec nisi euismod, dictum rhoncus leo. Maecenas sit amet aliquet massa. Praesent eu elementum mauris. Phasellus tincidunt augue sit amet metus tristique vulputate. Duis auctor eu velit et pharetra. Ut ut tortor et mi imperdiet rutrum. Nam pellentesque felis eget elit ultrices, at vulputate sem rhoncus. Curabitur adipiscing sapien velit, eget lobortis velit ullamcorper ut. Donec et tempor nulla. Proin ut velit sed erat porttitor pellentesque vitae non arcu. Mauris ut justo non risus commodo pretium.

Morbi sed sapien in nisi facilisis pellentesque. Nunc id arcu consectetur magna commodo posuere. Quisque malesuada quis libero nec facilisis. Cras pharetra sem ac ipsum faucibus, eu molestie erat fermentum. Proin ac magna tincidunt nisl tempus iaculis id ac eros. Fusce ullamcorper porttitor purus. Cras iaculis ullamcorper lacus vitae dapibus. Vestibulum nec suscipit risus. Donec quis leo vitae mi facilisis convallis. Pellentesque in ipsum nisi.Proin et placerat augue. Curabitur imperdiet tellus scelerisque, dapibus lectus nec, aliquet tortor. Etiam dignissim ornare orci, nec condimentum lectus. Praesent dictum nisl semper dui semper adipiscing. Ut in diam neque. Cras rutrum erat et nunc ultrices commodo at eget diam. Maecenas adipiscing tincidunt rutrum.
\include{./chapters/chap2}
\include{./chapters/chap3}
\include{./chapters/chap4}
\include{./chapters/chap5}

%-------------- Bibliography
\cleardoublepage
\phantomsection	\addcontentsline{toc}{chapter}{Bibliography}										
\printbibliography

%-------------- Apendicies
\cleardoublepage
\noappendicestocpagenum % Removes the number in the toc next to the 'appendices' entry. Each individual appendix still has a page number in the toc.
\begin{appendices}
\include{./appendices/appendix1}
\include{./appendices/appendix2}
\end{appendices}

\end{document}  

%---------------------------Notes-------------------------------

%-------------- 
% Floats and centreing for tables and figures \begin{table,scheme,figure} go in the body text, *then* include the file without any floats or centreing, closing with \end{table,scheme,figure} to ensure correct float placement. This is only if you are calling in your tables as seperate files, not leaving them in the text. Also note the american spelling of center (not the standard AU centre).

% e.g.

% Figure or Scheme (caption after)

% \begin{figure}
% \begin{center}
% \input{figures/name_of_figure.tex}
% \caption[Short caption for table of figures]{Long caption for text body}
% \end{center}
% \end{figure}

% Table (caption before)

% \begin{table}
% \begin{center}
% \caption[Short caption for table of tables]{Long caption for text body}
% \input{figures/name_of_figure.tex}
% \end{center}
% \end{table}

%-------------- 
% The difference between \input and \include is thus: \include creates an aux file for the included file, \input does not. \include is good for large files, like a chapter, since LaTeX won't reread the file if no changes have been made. \input is better for small inputs like data tables, since it doesn't create an aux file for the inputted file, it is read each and every time LaTeX is run. Additionally you *CAN* nest \input commands. you *CANNOT* nest \include commands.

%-------------- 








