\documentclass[12pt,oneside]{article}
\usepackage{geometry}
	\geometry{a4paper,inner=2cm, outer=3cm, top=3cm, bottom=3cm}

\usepackage[bitstream-charter]{mathdesign}
\usepackage[T1]{fontenc}

\usepackage[usenames, dvipsnames]{color}
\usepackage{titlesec}
    \titleformat{\section}
        {\color{RoyalBlue}\normalfont\Large\bfseries}
        {\color{RoyalBlue}\thesection}{1em}{}

\usepackage{pdfpages}
\usepackage{hyperref}
\makeatletter
\renewcommand*\l@section{\@dottedtocline{1}{1.5em}{2.3em}} % dots in the table of contents
\makeatother
\newcommand{\oporcom}[1]{\texttt{\color{RoyalBlue}#1}} % simplified the highlighting of options and commands in text
\setcounter{secnumdepth}{0}

\begin{document}
    
\title{\textbf{The \texttt{uowthesistitlepage} Package\\
Redefinition of \texttt{\textbackslash{}maketitle} to Create a Title Page for a Thesis at the University of Wollongong}}
\author{Thomas M. Griffiths}
\date{Released \today, Version 1.0}

\maketitle

\tableofcontents

\section{About}
This package redefines the \verb+\maketitle+ command for LaTeX documents and generates a title page compliant with the UOW branding guidelines. Use this package with the book class to typeset your thesis. The package has a number of options that change the wording of the title page according to: whether your thesis is part of, or the entire component of assessable work for your degree; the number of supervisors you have, if any. This package \emph{is not} a complete thesis template or document class, it only typesets the title page. The default options when the package is loaded are: \oporcom{honours} and \oporcom{onesupervisor}

\section{Legal}
This work is copyright (CC BY-NC-SA 3.0 AU) 2013 by T. M. Griffiths under the creative commons licence (attribution, non-comercial, share alike). More information can be found here: \url{http://creativecommons.org/licenses/by-nc-sa/3.0/au/}. 

This work may be distributed and/or modified under the conditions of the LaTeX Project Public License version 1.3c which can be found at: \url{http://www.latex-project.org/lppl/lppl-1-3c.txt}.

The crest of the University of Wollongong is copyright and the property of the University of Wollongong. As the core identifier of the university its use is governed by the university's brand and visual identity guidelines which can be found at: \url{http://www.uow.edu.au/about/brand/uowlogo/index.html}

\section{Usage}
If you are familiar with LaTeX using this package is very simple, place the \texttt{uowthesistitlepage} package in your document's preamble:
\begin{verbatim}
\usepackage[]{uowthesistitlepage} 
\end{verbatim}
The standard parts of the \verb+\maketitle+ command still apply. The title, author and date fields are filled in as normal, the package uses them in the title page.
\begin{verbatim}
\begin{document}
\title{A Pretty Swish Title} 
\author{Average J. Blow}
\date{Month Year}
\end{verbatim}
The new fields from the uowthesistitlepage package are\oporcom{\textbackslash{}degree}, \oporcom{\textbackslash{}school} and \oporcom{\textbackslash{}supervisor}. They're fairly self-explanatory and you use them just as you would the standard macros for the regular \verb+\maketitle+ command.  
\begin{verbatim}
\degree{That Degree You've Been Studying} 
\school{Your School} 
\supervisor{Supervisor 1, Supervisor 2 \& Supervisor 3 etc.}
\end{verbatim}
You need to ensure a copy of the logo named \texttt{uow\textunderscore{}logo} is in the working directory where you're compiling your thesis, if you downloaded this style from the UOW website there will be one in the zipped package. When you want to typeset the title page simply use \verb+\maketitle+ after all the defined fields as you usually would.

\section{Package Options and Macros}
\subsection*{Macros}
\begin{description}
    \item[\oporcom{\textbackslash{}author}]
    The author of the work, in this case you. For the your thesis you should use your full name and any previous degrees, for example: Average Joe Blogs BSc Hon.
    
    \item[\oporcom{\textbackslash{}date}]
    The date of your submission, usually `Month Year'. Alternatively, you can use the \oporcom{\textbackslash{}today} macro, and it will print the date in the format month dd, yyyy.
    
    \item[\oporcom{\textbackslash{}degree}]
    The degree for which you are submitting this work. Write the degree out in full; Doctor of Philosophy, Master of Science, Bachelor of Arts Honours etc.
    
    \item[\oporcom{\textbackslash{}graphic}]
    This macro is used internally to load the UOW Graphic. It has no immediate use in the body of your document and can be ignored unless you feel like cracking open the style file and tinkering. For the graphic to load properly in the title page there must be an image file in the working directory named \texttt{uow\textunderscore{}logo}. It can be used in the body of the document to load an alternative graphic if that is desired, be warned that determining the size in this case is up to you.
    The usage is a bit odd note the nested \verb+\includegraphics+ command:
    
    \oporcom{\textbackslash{}graphic}\texttt{\{\textbackslash{}includegraphics\{yourgraphic\}\}} 
    
    \item[\oporcom{\textbackslash{}requirement}]
    Similar to the \oporcom{\textbackslash{}graphic} macro, this macro is for internal use. It is manipulated by the degree options (\oporcom{phd}, \oporcom{honours} etc.) in the preamble, and determines what is printed in the requirement section of the title page. The default is the \oporcom{honours} option.
    
    \item[\oporcom{\textbackslash{}school}]
    The school within the university with which you are associated. If you are not associated with a school you can use the \oporcom{noschool} option in the preamble to disable the school under the UOW logo. The \oporcom{\textbackslash{}school} command will still work with the \oporcom{noschool} option, so you can specify another affiliation, for example The Australian Institute for Innovative Materials (AIIM).
    
    \item[\oporcom{\textbackslash{}supervisor}]
    Prints your supervisor(s) to the title page. List all your supervisors and their titles in the one set of enclosing braces (\texttt{\{ \}}), separated by commas with an ampersand (\&) before the final supervisor if you have multiple supervisors. To ensure the correct pluralisation of `supervisors' or the removal of the field if there are none, the number of supervisors needs to be defined as an option when you load the package in the preamble. The default option for supervisors is \oporcom{onesupervisor}
    
    \item[\oporcom{\textbackslash{}title}]
    Self explanatory, it's the title of your thesis
\end{description}

\subsection*{Options}
\begin{description}
    \item[\oporcom{honours}]
    One of the degree options and the default. This option modifies the wording of the thesis requirement so that it is reflective of an honours degree: `This thesis is presented as part of the requirement for the'.
    
    \item[\oporcom{honors}]
    Identical to \oporcom{honours}, just the alternative u-less spelling. See above.
    
    \item[\oporcom{phd}]
    One of the degree options. This option modifies the wording of the thesis requirement so that it is reflective of a Doctoral degree: `This thesis is presented as required for the'.
    
    \item[\oporcom{mastersbyresearch}]
    One of the degree options, the same as \oporcom{phd}. This option modifies the wording of the thesis requirement so that it is reflective of a Masters by Research: `This thesis is presented as required for the'.
    
    \item[\oporcom{multiplesupervisors}]
    Makes the supervisor field plural, printing `Supervisors'
    
    \item[\oporcom{noschool}]
    Removes the text underneath the UOW stating your school affiliation. While it removes the `School of' text it doesn't deactivate the \oporcom{\textbackslash{}school} command completely it prints an empty space instead. The \oporcom{\textbackslash{}school} command will still works with the \oporcom{noschool} option, so that you can specify another affiliation.
    
    \item[\oporcom{nosupervisor}]
    Removes the supervisor field from the title page completely. Incorporated mainly to match the Word template, which, oddly is missing a supervisor field.
    
    \item[\oporcom{onesupervisor}]
    Makes the supervisor field singular, printing `Supervisor' This option is the default.
\end{description}

\section{Change Log}
If you spot any errors or bugs, or alternately you have any requests for an addition let me know.
\subsubsection*{Version 1.0, 2013-23-06, tmgriffiths}
\begin{itemize}
    \item Implimented \oporcom{phd}, \oporcom{mastersbyresearch}, the various \oporcom{supervisor} based, and \oporcom{noschool} options.
    \item Annotated code in sty file.
    \item Wrote documentation.
\end{itemize}

\section{Example Title Pages}
In order on the following pages they are:
\begin{enumerate}
    \item \oporcom{onesupervisor,honours} \emph{default}
    \item \oporcom{multiplesupervisors,phd}
    \item \oporcom{nosupervisor,phd}
    \item \oporcom{noschool,honours}
    \item Microsoft Word 2010 template for comparison.
\end{enumerate}
\includepdf{onesupervisor.pdf}
\includepdf{multiplesupervisors-phd.pdf}
\includepdf{nosupervisor-phd.pdf}
\includepdf{noschool.pdf}
\includepdf{word.pdf}
\end{document}